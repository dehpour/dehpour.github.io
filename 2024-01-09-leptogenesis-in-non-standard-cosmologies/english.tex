\documentclass[a4paper]{book}
\usepackage[top=35mm, bottom=25mm, left=35mm, right=25mm]{geometry}
\usepackage[fontsize=12]{scrextend}
\usepackage{emptypage}
\usepackage{graphicx}
\usepackage[colorlinks]{hyperref}
\usepackage[sort&compress,numbers]{natbib}
\usepackage{bibentry}
\nobibliography*

\author{}
\title{}
\date{}

\begin{document}
	
	\pagenumbering{roman}
	
	\newgeometry{centering}
	\begin{titlepage}
		\thispagestyle{empty}
		\centering
		{\includegraphics[height=3cm]{./logo}\\Shahid Beheshti University\\Department of Physics \par}
		\vspace{1cm}
		{A Thesis Submitted in Partial Fulfillment of the Requirements for\\the Degree of Master of Science in Particle Physics and Field Theory \par}
		\vspace{1cm}
		{\huge Baryogenesis through leptogenesis \\in non-standard cosmologies \par}
		\vspace{1cm}
		{\large Mehran Dehpour\par}
		\vfill
		{\large January 2024\par}
	\end{titlepage}
	\restoregeometry
	
	\chapter*{Declaration of Authorship}
	\addcontentsline{toc}{chapter}{Declaration of Authorship}
	Mehran Dehpour hereby affirms that the thesis titled ``Baryogenesis through leptogenesis in non-standard cosmologies'' and the research presented therein are entirely his original study. In instances where he has referenced the published works of others, he has ensured to provide proper attribution. 
	The thesis has been constructed upon the following papers:
	\noindent
	\begin{itemize}
		\item \bibentry{Dehpour:2023dfo}
		\item \bibentry{Dehpour:2023wyy}
	\end{itemize}
	
	\chapter*{Acknowledgment}
	\addcontentsline{toc}{chapter}{Acknowledgment}
	This thesis is the result of research and learning on the subject of matter asymmetry, under the supervising of Siamak Sadat Gousheh and advising of Saeed Abbaslu. I am also grateful to Yasman Farzan and Pouya Bakhti for their guidance on neutrino physics. Therefore, I extend my sincerest appreciation to each of these individuals.
	
	I would like to express my deepest gratitude to my family, Sahar Safari and her family for their companionship throughout my undergraduate and graduate studies. Without their invaluable help and support, this thesis would not have been possible, and I am deeply grateful for their contributions in this regard.
	
	\chapter*{Abstract}
	\addcontentsline{toc}{chapter}{Abstract}
	{It is usually assumed that the Universe was produced initially without matter asymmetry or that the initial matter asymmetry was washed away by inflation. This implies that after the inflation, for every particle there was a corresponding antiparticle. One might then expect that this would result in the total annihilation of matter and antimatter as the temperature decreased, leading to our non-existence. However, fortunately, there is an excess of matter over antimatter which, nonetheless, is a problem that begs an explanation. Some researchers try to solve this problem by extending the standard model of particle physics. Some believe that the answer may lie in the neutrinos. The introduction of the sterile neutrinos, in conjunction with the seesaw mechanism, presents a viable explanation for the nonzero neutrino masses, which is beyond the standard model. In this framework, the mechanism of leptogenesis can also help address the matter asymmetry problem. However, the leptogenesis scenario, which relies on these components, has its drawbacks, such as the necessity for a large mass scale. It can lead to gravitino overproduction, in conflict with the supersymmetric models, and make the model untestable because of its inaccessible energy. In this work, we explore two methods for achieving low-scale leptogenesis through non-standard cosmologies. First, as we know, conventional statistical mechanics is not universal, and here we concentrate on the effects of Tsallis nonextensive statistical mechanics in the early Universe. Second, as we do not have signatures of isotropy before the big bang nucleosynthesis, we forsake the isotropic cosmological principle for a Bianchi type-I metric in the early Universe. We show that the use of nonextensive statistical mechanics can affect the production of baryon asymmetry in thermal leptogenesis by modifying the equilibrium abundance of particles, decay, and washout parameters. َAlso, our results show that for specific values of the anisotropy, the modified thermal leptogenesis can generate more baryon asymmetry than the standard one. In this way, our findings suggest that these approaches can facilitate low-scale leptogenesis. \par}
	\noindent
	Keywords: baryogenesis; thermal leptogenesis; nonextensive Tsallis statistical mechanics; Bianchi type-I.
	
	\bibliographystyle{JHEP}
	\bibliography{biblio}
	
\end{document}